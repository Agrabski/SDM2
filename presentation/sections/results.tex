\section{Wyniki}

\begin{frame}
	\frametitle{Prędkość}

	\begin{itemize}
		\item Zaimplementowany kompilator jest bardzo wolny
		\item 10 minut, aby skompilować ~200 linii kodu\begin{itemize}
			\item Duża część kodu kompilatora jest interpretowana
			\item Interpreter używa nieefektywnych struktur danych
			\item Implementacja interpretera jest bardzo prosta
			\item Używany parser jest generyczny
		\end{itemize}
	\end{itemize}

\end{frame}

\begin{frame}[fragile]
	\frametitle{Wykonywanie metaprogramowania}

	\begin{lstlisting}
fn print<T>(object: T*) {
	if(isPrimitive(typeof(T)))
		printPrimitive(object);
	else
	{
		for(field : typeof(T).fields)
		{
			printPrimitive(field.name);
			printPrimitive(" value is: ");
			print<field.type>(field.getValue(object));
		}
	}
}
	\end{lstlisting}

\end{frame}

\begin{frame}
	\frametitle{Wykonywanie metaprogramowania}

\begin{itemize}
	\item Wywołanie metafunkcji wygląda tak jak zwykłej procedury
	\item Wywoływanie niektórych funkcji nie ma sensu w trakcie kompilacji, nawet jeśli jest to możliwe\begin{itemize}
		\item Manipulacja plikami
		\item Interakcja z użytkownikiem
	\end{itemize}
	\item Śledzenie czy zmienna pozostaje stała jest trudne
\end{itemize}

\end{frame}

\begin{frame}
	\frametitle{Wnioski: nowe możliwości}

	\begin{itemize}
		\item W pracy zaprezentowano kilka praktycznych aplikacji proponowanych mechanizmów\begin{itemize}
			\item Generowanie plików nagłówkowych dla języka C
			\item Adnotacja no discard
			\item Automatyczna generacja typów wyliczeniowych typu flagi (w czasie realizacji)
		\end{itemize}
		\item C-=-1 dużo łatwiej rozszerzyć niż inne, podobne języki
		\item Biblioteki w C-=-1 łatwiej użyć z innego języka niż na przykład C++
	\end{itemize}

\end{frame}

\begin{frame}
	\frametitle{Wnioski: system typów}

	\begin{itemize}
		\item System typów jest wzorowany na C++/C\#\begin{itemize}
			\item Generyki/szablony dodane po fakcie
			\item Nie korzysta z pełnych możliwości współczesnej teorii
		\end{itemize}
		\item Adnotacje powinny być częścią systemu typów\begin{itemize}
			\item Implementacja atrybutu const i pochodnych
		\end{itemize}
		\item Relacja pomiędzy adnotacjami a generykami wymaga więcej pracy
	\end{itemize}

\end{frame}

\begin{frame}[fragile]
	\frametitle{Wnioski: system typów}

	\begin{columns}
		\begin{column}{0.5\textwidth}
			\begin{lstlisting}[caption={const w C-=-1}]
fn main() -> usize
{
 [const()] let x = 3;
 let list = List<usize*>();
 list.push(&x);
 [const()] let pointer = &x;
}
			\end{lstlisting}
		\end{column}
		\begin{column}{0.5\textwidth}
			\begin{lstlisting}[caption={const w C++}]
int main()
{
 auto x = 3;
 auto list = std::vector<int const*>();
 list.push_back(&x);
 auto const* pointer = &x;
}

			\end{lstlisting}
		\end{column}
	\end{columns}
\end{frame}



\begin{frame}
	\frametitle{Wnioski}

	\begin{itemize}
		\item Celem pracy było zbadanie jak zaproponowane mechanizmy wpływają na programowanie\begin{itemize}%todo: wording
			\item Z braku czasu, zaimplementowano minimalny podzbiór języka
			\item Badania zaproponowanego języka są raczej powierzchowne
		\end{itemize}
		\item To co zostało przetestowane, wskazuje nowe kierunki badań\begin{itemize}
			\item Jaka powinna być relacja między atrybutami a systemem typów?
			\item Czy kompilator może działać z akceptowalną prędkością?
		\end{itemize}
		\item Pozostają dalsze badania nad wpływem tej formy metaprogramowania na tworzenia aplikacji.
	\end{itemize}

\end{frame}
