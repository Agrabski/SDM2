\section{C-=-1}

\begin{frame}
	\frametitle{Motywacja}

	\begin{itemize}
		\item Metaprogramowanie w C\# i Javie jest potężnym narzędziem
		\item Wiążę się to z dużym kosztem w czasie uruchomienia
		\item Wymaga środowiska uruchomieniowego i informacji o strukturze programu w pliku wykonywalnym
		\item W wielu kontekstach to nie jest akceptowalne
	\end{itemize}
\end{frame}

\begin{frame}[fragile]
	\frametitle{Motywacja}
	\begin{columns}
		\begin{column}{0.5\textwidth}
\begin{itemize}
	\item Kod źródłowy programu powinien być poprawny, jasny, niezawodny i zwięzły
	\item Niektóre fragmenty kodu są zależne od innych\begin{itemize}
		\item Serializacja
	\end{itemize}
	\item Lepiej unikać takich zależności
	\item Metaprogramowanie to sposób na poprawę niezawodności i zwięzłości
\end{itemize}
		\end{column}
		\begin{column}{0.5\textwidth}  %%<--- here
			\begin{center}
\begin{lstlisting}
struct person {
	std::string name;
	int yearOfBirth;
	int height;
};
void print(
  std::ostream&s,
  person const& p) {
  s << p.name << '\n'
    << p.yearOfBirth << '\n'
    << p.height << '\n';
}
\end{lstlisting}
			 \end{center}
		\end{column}
		\end{columns}

\end{frame}

\begin{frame}
	\frametitle{Statyczne metaprogramowanie}

	\begin{itemize}
		\item Mechanizmy metaprogramistyczne zwykle działają w trakcie uruchomienia\begin{itemize}
			\item Java
			\item C\#
		\end{itemize}
		\item Statyczne metaprogramowanie w ostatnich latach się mocno rozwija\begin{itemize}
			\item Rust
			\item Szablony w C++
			\item Generatory kodu w C\#
		\end{itemize}
		\item Przez dodanie generatorów kodu do C\#, serializacja do Json przyśpieszyła o 30\%
	\end{itemize}

\end{frame}

\begin{frame}
	\frametitle{Cel języka}

	\begin{itemize}
		\item Wprowadzenie metaprogramowania do języków niskopoziomowych
		\item Umożliwienie wykonania dowolnego kodu w trakcie kompilacji\begin{itemize}
			\item Wyrażenia oczekujące stałych
			\item Optymalizacje
		\end{itemize}
	\end{itemize}

\end{frame}

\begin{frame}
	\frametitle{Atrybuty}

	\begin{itemize}
		\item Mechanizm podobny do Atrybutów z C\#
		\item Adnotacje dla typów, funkcji, zmiennych itp.
		\item Nowym elementem są funkcje reagujące na użycie adnotowanego obiektu
		\item Wewnątrz tej funkcji, atrybut może badać i modyfikować kod programu
	\end{itemize}

\end{frame}

\begin{frame}[fragile]
	\frametitle{Atrybuty}

	\begin{lstlisting}
public att<function> NoDiscard
{
	public fn attach(f: functionDescriptor)
	{}

	public fn onCall(call: functionCallExpression*)
	{
		if(call._parentStatment != null<IInstruction>())
			raiseError(&(call._pointerToSource), "Return value of a no-discard function is not used", 123);
	}
}
		
	\end{lstlisting}

\end{frame}

\begin{frame}[fragile]
	\frametitle{Atrybuty}

	\begin{lstlisting}
[noDiscard()]
fn noDiscardFunction() -> usize;

fn main() -> usize
{
  noDiscardFunction(); // error 123: Return value of
                       // a no-discard function is not used
  let x = noDiscardFunction();     // ok
  let y = x + noDiscardFunction(); // ok
  return noDiscardFunction();      // ok
}
		
	\end{lstlisting}

\end{frame}


